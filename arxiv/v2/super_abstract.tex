The "mind-brain supervenience" conjecture suggests that all mental properties are derived from the physical properties of the brain. To address the question of whether the mind supervenes on the brain, we frame a supervenience hypothesis in rigorous statistical terms. Specifically, we propose a modified version of supervenience (called epsilon-supervenience) that is amenable to experimental investigation and statistical analysis. To illustrate this approach, we perform a thought experiment that illustrates how the probabilistic theory of pattern recognition can be used to make a one-sided determination of epsilon-supervenience. The physical property of the brain employed in this analysis is the graph describing brain connectivity (i.e., the brain-graph or connectome). epsilon-supervenience allows us to determine whether a particular mental property can be inferred from one's connectome to within any given positive misclassification rate, regardless of the relationship between the two. This may provide motivation for cross-disciplinary research between neuroscientists and statisticians.
% request for removal from telemarketing lists

\documentclass[10pt]{letter}

\oddsidemargin=.2in
\evensidemargin=.2in
\textwidth=6in
\topmargin=-1.2in
\textheight=10.2in


\name{	Joshua T.\ Vogelstein, PhD. \\ 
		Johns Hopkins University \\
		% Ph.D.\ Candidate in the Dept.\ of Neuro. \\
		% Postdoc., 
		Dept. Applied Math.\ \& Stat's \\
		3400 N.\ Charles St.\ \\
		Whitehead 105C \\
		Baltimore, MD 21210 \\
      	c: 443.858.9911 \\
		e: joshuav@jhu.edu
		}

\date{\today}

\begin{document}


\begin{letter}{	%Prof.\ Gerald Edelman, M.D.\, Ph.D.\ \\
               	% Chairman,The Scripps Research Institute \\
               	% 10550 N.\ Torrey Pines Rd.\ \\
               	% La Jolla, CA  92037-1000
}   
                           
\opening{Dear Editorial Board:}

On behalf of myself and my coauthors, I submit for your review a copy of our manuscript entitled, ``Are mental properties supervenient on brain properties?'' We hope that you will consider publishing this manuscript in Nature Scientific Reports.

At the core of this paper is a thought experiment in which we mathematically prove that if the graph corresponding to brain connectivity (i.e., the \emph{connectome}) is statistically related to a particular mental property (e.g., propensity for mathematics), one could build a classifier to determine whether any individual's brain exhibits that property with an arbitrarily small misclassification rate. The proof follows directly from the probabilistic theory of pattern recognition and is made possible by our novel exposition of universal consistency for a $k_n$-nearest neighbor classifer on graphs. In addition to these theoretical results, the manuscript describes a simulation that suggests how one could actually build such a classifier for the ``brain'' of \emph{Caenorhabditis elegans} using today's technology.

% Our results have many implications, some philosophical, some statistical, and some pragmatic. First, the philosophical implications: The relationship between the mind and brain has been investigated for millennia, using tools ranging from abstract arguments to microstimulation of individual neurons. However, until now, no one (to our knowledge) has proposed a framework that allows for empirical investigation of this relationship at the scale of the connectome. Such a framework will undoubtedly be of value as the multiple ongoing large-scale efforts to collect a connectome (cf.\ NIH Human Connectome Project) begin paying dividends.
% 
% In addition to the philosophical implications, our work has significant statistical implications. 
% Whereas several groups have recently proposed algorithms for classifying graphs, there exists very little theoretical work on the limitations of these tools. In contrast, our work is founded on a constructive proof of a universally consistent classifier on graphs, the limitations of which are both known and well-understood. Furthermore, the proof is shown to be only one-sided, meaning that we can never gain confidence in the negation of the null hypothesis. Although we only apply this classifier to simulated brain-derived graphs, the methods are applicable to many other kinds of graph data. The recent emergence of networks and graphs as the preferred representations of structured relationships in multiple scientific disciplines suggests that the development of statistical tools for rigorous analysis of network data is a pressing need for the scientific community at large.
% 
% Finally, the results described in this manuscript have pragmatic implications. Studies relating brain structure to brain function are being published with increasing frequency due to the increasing availability of diffusion-weighted magnetic resonance imaging and other connectivity-imaging technologies. However, without sophisticated statistical tools for the analysis of such data, to-date the findings have (mostly) been limited to linear correlations between scalar measures of connectivity and the brain function under investigation. The graph-theoretic approach proposed here allows investigators to utilize the entirety of the information obtained from such measures, rather than using simple descriptive statistics.


We believe that the work described in this manuscript makes significant contributions in several contemporary fields of inquiry, including neuroscience, philosophy, and statistics.  It is for these reasons that we believe it is particularly well suited for Nature Scientific Reports.  
% and is therefore suited for a general-audience journal such as Nature Scientific Reports. %Your interest in the philosophical underpinnings of neuroscience, as well as in inferring causal relationships from neural data and generating large-scale simulations of neural populations suggests to us that this manuscript would of interest to you. 

As a scientific reviewer, we suggest Prof. Giorgio Ascoli (ascoli@gmu.edu), as he is a scholar in neuroscience and neuroinformatics, with a solid background in quantitative studies, with interests in whole brain analysis.  Please exclude Prof. Olaf Sporns, as our work and his closely overlap, and he has seen previous versions of this manuscript. 

Please do not hesitate to contact us with any questions or comments.

\closing{Respectfully,}



\end{letter}


\end{document}

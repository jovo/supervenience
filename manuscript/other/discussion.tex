\paragraph{Multiple Realizability}

$B \in \mB_i$

\paragraph{delta varepsilon}

$M \overset{\delta,\varepsilon}{\sim} \mB \Leftrightarrow \exists f(\cdot), \; B \in \mB$ s.t. $P[d(f^\ast(B)\neq M)< \delta]< \varepsilon$

\paragraph{superdupervenience}

from stanford encyclopedia:

Supervenience claims, by themselves, do nothing more than state that certain patterns of property (or fact) variation hold. They are silent about why those patterns hold, and about the precise nature of the dependency involved (see Kim 1993, 167; 1998, 9-15; Blackburn 1984, 186; Schiffer 1987, 153-154; and McGinn 1993, 57). But supervenience theses are not plausibly brute, that is, unexplainable. It is natural to look further, and to try to explain why A-properties supervene on B-properties. When such supervenience is explainable, there is ÔsuperdupervenienceÕ (a term coined by William Lycan; see also Schiffer 1987; Horgan 1993; and Wilson 1999).

\newpage
\paragraph{related work}

\paragraph{math}


graph classification papers: \cite{KashimaInokuchi02, KudoMatsumoto04, Emmert-StreibKilian05, PredictingStructuredData07, BunkeRiesen08, SaigoTsuda09, TrentinIorio09, ZaslavskiyVert08, BunkeRiesen08, RiesenBunke09}

T. Gaertner, A survey of kernels for structured data, SIGKDD Exploration Newsletter 5 (1) (2003) 49Ð58.

M. Gori, G. Monfardini, F. Scarselli, A new model for learning in graph domains, in: Proceedings of IJCNN-05, August 2005. in 1990, and his Ph.D. in Automation and Information

A. Sperduti, A. Starita, Supervised neural networks for the classification of structures, IEEE Transactions on Neural Networks 8 (3) (1997) 714-735.

P.A. Flach, N. Lachiche, Naive Bayesian classification of structured data, Machine Learning 57 (3) (2004) 233–269. 

D.J. Hand, K. Yu, Idiot’s bayes—not so stupid after all?, International Statistical Review 69 (2001) 385–398.

B.J. Jain, P. Geibel, F. Wysotzski.  SVM learning with the Schur-Hadamard inner product for graphs Neurocomputing (2005) 64:93-105

\paragraph{DTI}

book on DTI: \cite{DTIbook09}

higher res technology, DSI: \cite{AlexanderPark01}

q-ball imaging (between DTI and DSI): \cite{Tuch03}


\paragraph{biology}

human connectome: \cite{SpornsKotter05, Seung09}

neuron level methods papers: \cite{DenkHorstmann04, BriggmanDenk06, LuLichtman09}

neuroscience finding papers will depend somewhat on the specific covariates we get, as there are many possible papers here.  the DTI book has a few in it.

anatomy relationship to mental properties: \cite{HutchinsonSchlaug03}

\paragraph{c elegans}

got the graph: \cite{WhiteBrenner86}
looked at ``network statistics'' (like in-degree and out-degree distributions), might be useful for simulations: \cite{ReiglChklovski04}

